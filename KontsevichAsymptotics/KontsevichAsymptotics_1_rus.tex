\documentclass[12pt, a4paper, titlepage]{article}
\usepackage[T2A]{fontenc}
\usepackage[utf8]{inputenc}
\usepackage[english, russian]{babel}
\usepackage{amssymb,amsfonts,amsmath,mathtext,cite,enumerate,float, bbm}
\usepackage{tikz}
\usepackage{verbatim}

\pagestyle{plain}
\righthyphenmin = 2

\usepackage{geometry}
\geometry{left=3cm}
\geometry{right=1cm}
\geometry{top=3cm}
\geometry{bottom=3cm}

\usepackage{amsthm} \makeatletter
\renewcommand*{\@seccntformat}[1]{ \csname the#1\endcsname.\enskip
}
\makeatother

\newcommand{\startvec}{\begin{pmatrix}
                        1\\0
                       \end{pmatrix}}
\newcommand{\nullvec}{\begin{pmatrix}
                        0\\0
                       \end{pmatrix}}

\begin{document}

\newtheorem{theorem}{Теорема}
\newtheorem{statement}{Утверждение}
\newtheorem{proposition}{Предложение}
\newtheorem{definition}{Определение}
\newtheorem{lem}{Лемма}
\newtheorem{example}{Пример}
\newtheorem{remark}{Замечание}

\section{Асимптотика инвариантов Васильева высших порядков}
\begin{comment}
Пусть $Z_t, t \ge 1$ --- двумерное броуновское движение, выходящее из точки $\startvec$:
$$Z(1)=\startvec.$$
Введём, следуя~\cite{BertoinWerner}, процесс $X_t$ с помощью экспоненциальной замены времени
для $Z_t$:
$$X(t)=e^{-t/2}Z(e^t), t \ge 0.$$
Тогда $X(t)$ --- двумерный процесс Орнштейна-Уленбека. \\
Пусть $R(t)=|X(t)|, H(t)=\int\limits_0^t \frac{ds}{R_s^2}.$
\end{comment}

\begin{comment}
Можно показать, что конечномерные распределения процесса
$\frac{H(tT)}{T^2}$ сходятся при $T\to\infty$ к конечномерным
распределениям процесса $S,$
где \\
$$S(t)=\int\limits_0^{\sigma(t)}\mathbbm{1}_{w_s \le 0}ds, t \ge 0,$$
$w$ --- одномерное броуновское движение с $w(0)=0,$ \\
$l_t$ --- локальное время $w$ в нуле, \\
$\sigma(t)=\inf\{s > 0: l_s >t\}.$
\end{comment}

\begin{comment}
\begin{statement}
 Пусть $L(t)=\int\limits_0^t H_1(u)dH_2(u).$
Тогда 
$$\frac{L(t)}{t^2}\xrightarrow[t\to\infty]{d}\int\limits_0^1 S_1(u)dS_2(u).$$
\end{statement}
\begin{proof}
 Процессы $H_1, H_2$ --- возрастающие с вероятностью $1.$
Для любого $n\in \mathbb{N}$ имеем 
\begin{multline*}
 \frac{1}{t^4}\sum\limits_{k=0}^{n-1}H_1\left(\frac{k}{n}t\right)
 \left(H_2\left(\frac{k+1}{n}t\right)-H_2\left(\frac{k}{n}t\right)\right) \le
 \frac{1}{t^4}\int\limits_0^t H_1(u)dH_2(u) 
 \le 
 \\
 \le
 \frac{1}{t^4}\sum\limits_{k=0}^{n-1}H_1\left(\frac{k+1}{n}t\right)
 \left(H_2\left(\frac{k+1}{n}t\right)-H_2\left(\frac{k}{n}t\right)\right).
\end{multline*}
Заметим, что в силу сходимости конечномерных распределений процессов $H_1, H_2$, при фиксированном $n$
имеем
 \begin{multline*}
 \frac{1}{t^4}\sum\limits_{k=0}^{n-1}H_1\left(\frac{k}{n}t\right)
 \left(H_2\left(\frac{k+1}{n}t\right)-H_2\left(\frac{k}{n}t\right)\right)
 \xrightarrow[t\to\infty]{d}
  \sum\limits_{k=0}^{n-1}S_1\left(\frac{k}{n}\right)
 \left(S_2\left(\frac{k+1}{n}\right)-S_2\left(\frac{k}{n}\right)\right).
\end{multline*}

С другой стороны,
\begin{multline*}
 \frac{1}{t^4}\sum\limits_{k=0}^{n-1}H_1\left(\frac{k+1}{n}t\right)
 \left(H_2\left(\frac{k+1}{n}t\right)-H_2\left(\frac{k}{n}t\right)\right)-\\-
 \frac{1}{t^4}\sum\limits_{k=0}^{n-1}H_1\left(\frac{k}{n}t\right)
 \left(H_2\left(\frac{k+1}{n}t\right)-H_2\left(\frac{k}{n}t\right)\right)
 \xrightarrow[t\to\infty]{d} 
 \\
 \xrightarrow[t\to\infty]{d}
 \sum\limits_{k=0}^{n-1}\left(S_1\left(\frac{k+1}{n}\right)-S_1\left(\frac{k}{n}\right)\right)
 \left(S_2\left(\frac{k+1}{n}\right)-S_2\left(\frac{k}{n}\right)\right).
\end{multline*}
Заметим, что
$$
\sum\limits_{k=0}^{n-1}\left(S_1\left(\frac{k+1}{n}\right)-S_1\left(\frac{k}{n}\right)\right)
 \left(S_2\left(\frac{k+1}{n}\right)-S_2\left(\frac{k}{n}\right)\right)\xrightarrow[n\to\infty]{P}0.
$$
Действительно, 
\end{proof}

\begin{statement}
 Имеет место асимптотическое соотношение
 $$\frac{1}{T}\int\limits_0^T\alpha_{12}(s)ds\xrightarrow[T\to\infty]{d}$$
\end{statement}
\begin{proof}
 Сначала покажем, что $\frac{1}{T}\int\limits_0^T\alpha_{12}(s)ds$ имеет предел по распределению.
 Рассмотрим разность
\begin{multline*}
\frac{1}{T^2}\int\limits_0^T \alpha_{12}(s)ds-
 \frac{1}{T^2}\sum\limits_{k=0}^{n-1}\int\limits_{\frac{k}{n}T}^{\frac{k+1}{n}T}
  \frac{T}{n}\alpha_{12}(kT/n)=
 \sum\limits_{k=0}^{n-1}
 \int\limits_{\frac{k}{n}T}^{\frac{k+1}{n}T}\left(\alpha_{12}(s)-\alpha_{12}(kT/n)\right)ds
 \le \\ \le
 \frac{1}{Tn}\sum\limits_{k=0}^{n-1}\max\limits_{s\in [kT/n, (k+1)T/n]}|\alpha_{12}(s)-\alpha_{12}(kT/n)|.
\end{multline*}
 Вспомним, что $\alpha_{12}(s)=\theta_{12}(e^s).$
 Имеем
 $$<\theta_{12}>_t=2\int\limits_0^t\frac{ds}{|Z_{12}(s)|^2}.$$
Отсюда получаем
$$<\alpha_{12}>(t)=<\theta_{12}>(e^t)=2\int\limits_0^t\frac{ds}{R_{12}(s)^2}=H_{12}(t).$$
Итак, имеем
$$\alpha_{12}(t)=\beta_{12}(H_{12}(t))$$
для некоторого броуновского движения $\beta_{12}$, не зависящего от процесса $R_{12}.$
Поэтому
\begin{multline*}
 \frac{1}{Tn}\sum\limits_{k=0}^{n-1}\max\limits_{s\in [kT/n, (k+1)T/n]}|\alpha_{12}(s)-\alpha_{12}(kT/n)|
 \xrightarrow[T\to\infty]{d}\\
 \xrightarrow[T\to\infty]{d}
 \frac{1}{n}\sum\limits_{k=0}^{n-1}\max\limits_{s\in [S(k/n), S((k+1)/n)]}|\beta(s)-\beta(S(k/n))|.
\end{multline*}
\end{proof}

\begin{lem}\label{WeakConvergenceFirstLem}
Пусть набор случайных величин $\gamma_n(T) (T>0), \gamma_n$
 таков, что
\begin{itemize}
 \item $\forall n \,\gamma_n(T)\xrightarrow[T\to\infty]{d}\gamma_n;$
 \item $\gamma_n \xrightarrow[n\to\infty]{d} 0.$
\end{itemize}
Тогда 
$$\forall \varepsilon>0 \enskip \exists n_0 \enskip
\forall n \ge n_0 \enskip \exists T_0(n)\enskip \forall T \ge T_0(n) \colon
P(|\gamma_n(T)|>\varepsilon)<\varepsilon.$$
\end{lem}
\begin{proof}
  Фиксируем $\varepsilon>0.$
Выберем $n$ так, что 
$$\forall n \ge n_0 \enskip P(|\gamma_n|>\varepsilon)<\varepsilon.$$
Выберем при любом $n \ge n_0$ точку $\alpha_n$, такую, что
\begin{itemize}
 \item $\varepsilon<\alpha_n<2\varepsilon$;
 \item $\alpha_n$ --- точка непрерывности функции распределения случайной величины $|\gamma_n|.$
\end{itemize}
Тогда из слабой сходимости $\gamma_n(T)\xrightarrow[T\to\infty]{d}\gamma_n$
вытекает
$$P(|\gamma_n(T)|>\alpha_n)\to P(|\gamma_n|>\alpha_n), T\to\infty.$$ 
Но $P(|\gamma_n|>\alpha_n)<\varepsilon.$
Следовательно, существует $T_0(n)$:
$$\forall T \ge T_0(n) P(|\gamma_n(T)|>\alpha_n) \le \varepsilon.$$
Значит, при $n \ge n_0, T \ge T_0(n)$
$$P(|\gamma_n(T)|>2\varepsilon)\le P(|\gamma_n(T)|>\alpha_n)\le
\varepsilon<2\varepsilon.$$
Поскольку $\varepsilon$ произвольно, лемма доказана.
\end{proof}

\begin{lem}\label{WeakConvergenceSecondLem}
Пусть набор случайных величин $\gamma_n(T)\ge 0 (T>0), \gamma_n\ge 0$
и броуновских движений $\beta_{n,T}$
таков, что
\begin{itemize}
 \item $\forall n \,\gamma_n(T)\xrightarrow[T\to\infty]{d}\gamma_n;$
 \item $\gamma_n \xrightarrow[n\to\infty]{d} 0.$
\end{itemize}
Тогда
$$\forall \varepsilon>0 \enskip \exists n_0 \enskip
\forall n \ge n_0 \enskip \exists T_0(n)\enskip \forall T \ge T_0(n) \colon
P(\sup\limits_{0\le s \le \gamma_n(T)}|\beta_{n,T}(s)|>\varepsilon)<\varepsilon.$$
\end{lem}
\begin{proof}
 Фиксируем $\varepsilon>0$. Выберем $\delta, 0<\delta<\varepsilon$ так, что для винеровского
процесса $\beta$
$$P(\sup\limits_{0\le s \le \delta}|\beta(s)|>\varepsilon)<\varepsilon.$$
Используя лемму~\ref{WeakConvergenceFirstLem}, получаем:
$$\exists n_0 \enskip
\forall n \ge n_0 \enskip \exists T_0(n)\enskip \forall T \ge T_0(n) \colon
P(\gamma_n(T)>\delta)<\delta.$$
Отсюда при $n \ge n_0, T \ge T_0(n)$ имеем
$$P(\sup\limits_{0\le s \le \gamma_n(T)}|\beta_{n,T}(s)|>\varepsilon) \le 
P(\sup\limits_{0\le s \le \delta}|\beta_{n, T}(s)|>\varepsilon)+
P(\gamma_n(T)>\delta)<\varepsilon+\delta<2\varepsilon.$$
Поскольку $\varepsilon$ произвольно, лемма доказана.
\end{proof}
\end{comment}

\begin{lem}\label{WeakConvergenceThirdLem} 
 Пусть набор случайных величин $\xi(T), T>0$ таков, что
 имеются случайные величины $\eta_n(T), \gamma_n(T)\ge 0, \gamma_n\ge 0, \eta_n$
 такие, что
\begin{itemize}
 \item $\forall T\, |\xi(T)-\eta_n(T)| \le \gamma_n(T)$;
 \item $\forall n \,\eta_n(T)\xrightarrow[T\to\infty]{d}\eta_n$;
 \item $\forall \varepsilon>0 \enskip \exists n_0 \enskip
\forall n \ge n_0 \enskip \exists T_0(n)\enskip \forall T \ge T_0(n) \colon
P(\gamma_n(T)>\varepsilon)<\varepsilon.$
\end{itemize}
Тогда $\xi(T)$ имеет предел по распределению при $T\to\infty$:
$$\xi(T)\xrightarrow[n \to \infty]{d}\xi_0.$$
Более того, при этом  $\eta_n\xrightarrow[T\to\infty]{d}\xi_0.$
\end{lem}
\begin{proof}
 Фиксируем $\varepsilon>0$. Выберем $n_0$ и $T_0(n)$
 так, что при $n \ge n_0, T \ge T_0(n)$:
 $$P(\gamma_n(T)>\varepsilon)<\varepsilon.$$
 Тогда при $n \ge n_0, T \ge T_0(n)$:
 $$\mathbb{E}\frac{\gamma_n(T)}{1+\gamma_n(T)} \le 2\varepsilon.$$
 
 Пусть $\mu_{\xi}$---распределение случайной величины $\xi$,
 и пусть $\rho$ обозначает расстояние Вассерштейна нулевого порядка, 
 т.е. для случайных величин $\xi, \eta$
 $$\rho(\mu_{\xi}, \mu_{\eta})=\inf\limits_{\xi'\stackrel{d}{=}\xi, \eta'\stackrel{d}{=}\eta}
  \mathbb{E}\frac{|\xi'-\eta'|}{1+|\xi'-\eta'|}.$$
 Для каждого $n \ge n_0$ выберем $T_1(n)>T_0(n)$ так, чтобы при $T \ge T_1(n)$ выполнялось следующее условие:
  $$\rho(\mu_{\eta_n(T)}, \mu_{\eta_n})\le \varepsilon.$$
 В силу $T_1(n)>T_0(n)$, при $n \ge n_0, T \ge T_1(n)$ выполнено ещё и
 $$\mathbb{E}\frac{\gamma_n(T)}{1+\gamma_n(T)}\le 2\varepsilon.$$

 Имеем тогда при $n \ge n_0$, $T\ge T_1(n)$, учитывая неравенство $|\xi(T)-\eta_n(T)|\le \gamma_n(T)$:
$$\rho(\mu_{\xi(T)},\mu_{\eta_n(T)})\le \mathbb{E}\frac{|\xi(T)-\eta_n(T)|}{1+|\xi(T)-\eta_n(T)|}\le
\mathbb{E}\frac{|\gamma_n(T)|}{1+|\gamma_n(T)|}\le 2\varepsilon.$$
Далее, имеем при $n \ge n_0$, $T\ge T_1(n)$:
$$\rho(\mu_{\xi(T)},\mu_{\eta_n})\le\rho(\mu_{\xi(T)}, \mu_{\eta_n(T)})+\rho(\mu_{\eta_n(T)},\mu_{\eta_n})
\le 2\varepsilon+\varepsilon = 3\varepsilon.$$
Отсюда следует, что для любых $t_1, t_2\ge T_1(n_0)$ имеет место
$$\rho(\mu_{\xi(t_1)}, \mu_{\xi(t_2)})\le \rho(\mu_{\xi(t_1)}, \mu_{\eta_N})+
\rho(\mu_{\xi(t_2)}, \mu_{\eta_{n_0}})\le 6 \varepsilon.$$
Следовательно, в силу произвольности выбора $\varepsilon$, последовательность
$\{\mu_{\xi(T)}\}$ фундаментальна в метрике $\rho$ и имеет слабый предел:
$$\xi(T)\xrightarrow[T\to\infty]{d} \xi_0.$$
Осталось показать, что $\eta_n\xrightarrow[n\to\infty]{d}\xi_0.$
Выберем $T_2$ так, что при $T\ge T_2$:
$$\rho(\mu_{\xi(T)}, \mu{\xi_0})\le \varepsilon.$$
Имеем тогда при $n\ge N, T\ge \max\{T_1(n), T_2\}$:
$$\rho(\mu_{\xi_0}, \mu_{\eta_n})
\le 
\rho(\mu_{\xi_0}, \mu_{\xi(T)})+
\rho(\mu_{\xi(T)}, \mu_{\eta_n})
\le
\varepsilon+3\varepsilon=4\varepsilon.$$
Отсюда следует сходимость $\rho(\mu_{\eta_n}, \mu_{\xi_0}) \xrightarrow[n\to\infty]{}0,$
и доказательство окончено.
\end{proof}

\begin{comment}
\begin{lem}\label{WeakConvergenceFourthLem}
 Пусть набор случайных величин $\xi(T), T>0$ таков, что
 имеются случайные величины $\eta_n(T), \gamma_n(T) (T>0), \gamma_n, \eta_n$
 такие, что
\begin{itemize}
 \item $\forall T\, |\xi(T)-\eta_n(T)| \le \gamma_n(T)$;
 \item $\forall n \,\gamma_n(T)\xrightarrow[T\to\infty]{d}\gamma_n;$
 \item $\gamma_n \xrightarrow[n\to\infty]{d} 0;$
 \item $\forall n \,\eta_n(T)\xrightarrow[T\to\infty]{d}\eta_n.$
\end{itemize}
Тогда $\xi(T)$ имеет предел по распределению при $T\to\infty$:
$$\xi(T)\xrightarrow[n \to \infty]{d}\xi_0.$$
Более того, при этом  $\eta_n\xrightarrow[T\to\infty]{d}\xi_0.$
\end{lem}
\begin{proof}
 Сразу следует из лемм~\ref{WeakConvergenceFirstLem}
 и~\ref{WeakConvergenceThirdLem}.
\end{proof}


\begin{lem}\label{WeakConvergenceFifthLem}
 Пусть набор случайных величин $\xi(T), T>0$ таков, что
 имеются случайные величины $\eta_n(T), \zeta_n(T) \ge 0, \mu_n(T) \ge 0, \mu_n \ge 0, \eta_n$
 и винеровские процессы $\beta_{n, T}$ такие, что
\begin{itemize}
 \item $\forall T\, |\xi(T)-\eta_n(T)| \le |\beta_{n, T}(\zeta_n(T))|;$
 \item $\zeta_n(T) \le \mu_n(T);$
 \item $\forall n \,\mu_n(T)\xrightarrow[T\to\infty]{d}\mu_n;$
 \item $\mu_n \xrightarrow[n\to\infty]{d} 0;$
 \item $\forall n \,\eta_n(T)\xrightarrow[T\to\infty]{d}\eta_n.$
\end{itemize}
Тогда $\xi(T)$ имеет предел по распределению при $T\to\infty$:
$$\xi(T)\xrightarrow[n \to \infty]{d}\xi_0.$$
Более того, при этом  $\eta_n\xrightarrow[T\to\infty]{d}\xi_0.$
\end{lem}\begin{proof}
 Сразу следует из лемм~\ref{WeakConvergenceSecondLem}
 и~\ref{WeakConvergenceThirdLem}.
\end{proof}
\end{comment}

\begin{lem}\label{infiniteTreeLem}
 Пусть $\varepsilon>0$, $f\in \mathbb{D}[0,1]$ не имеет скачков величины больше $\varepsilon.$
 Тогда 
 $$\exists n_0 \forall n\ge n_0 \forall k, 0\le k \le 2^n-1 \colon 
 \sup\limits_{s,t \in [\frac{k}{2^n}, \frac{k+1}{2^n}]} |f(s)-f(t)|<\varepsilon.$$ 
\end{lem}
\begin{proof}
 Допустим, это не так. Тогда при каждом $n$ существуют отрезки разбиения,
 на которых $\sup\limits_{s,t \in [\frac{k}{2^n}, \frac{k+1}{2^n}]} |f(s)-f(t)| \ge \varepsilon.$
 Построим дерево, в котором вершинами будут такие отрезки. 
 При этом для вершины, соответствующей отрезку $[\frac{k}{2^n}, \frac{k+1}{2^n}]$,
 дочерними будут вершины, соответствующие отрезкам 
 $[\frac{k}{2^n}, \frac{2k+1}{2^{n+1}}]$ и
 $[\frac{2k+1}{2^{n+1}}, \frac{k+1}{2^n}]$ (если они входят в дерево).
 По предположению, наше дерево бесконечно. Значит, в нём есть бесконечная ветвь.
 В пересечении отрезки, соответствующие вершинам этой ветви, дают некоторую точку $x$. 
 Эта точка $x$ обладает тем свойством, что в любой её окрестности есть $s,t$, такие, что
 $|f(s)-f(t)| \ge \varepsilon$. Следовательно, скачок в точке $x$ не меньше $\varepsilon.$ 
\end{proof}

\begin{lem}\label{boundedJumpsNumberLem}
 Пусть $f\in \mathbb{D}[0,1]$, $\varepsilon>0$ задано. Тогда существует $K>0$, такое, что
 при всех достаточно больших $n$ ($n \ge n_0$) выполнено
 $$\sum\limits_{k=0}^{2^n-1}\mathbbm{1}_{\sup\limits_{s,t \in [\frac{k}{2^n}, \frac{k+1}{2^n}]}
|f(s)-f(t)|>\varepsilon} \le K,$$
причём при $n \ge n_0$ все отрезки $[\frac{k}{2^n}, \frac{k+1}{2^n}]$, для которых 
$\sup\limits_{s,t \in [\frac{k}{2^n}, \frac{k+1}{2^n}]}|f(s)-f(t)|>\varepsilon$,
содержатся среди отрезков, содержащих точки разрыва величины $\ge \varepsilon.$
\end{lem}
\begin{proof}
 Уберём из $f$ все скачки величины $\ge \varepsilon.$ Тогда лемма сводится к лемме~\ref{infiniteTreeLem}.
\end{proof}

\begin{lem}\label{simultaneousJumpsLem}
 Пусть $f, g \in \mathbb{D}[0,1]$ не имеют общих моментов скачков. Тогда
 $$\forall \varepsilon>0 \enskip \exists n_0 \enskip \forall n\ge n_0
 \sum\limits_{k=0}^{2^n-1}
 \mathbbm{1}_{\sup\limits_{s,t \in [\frac{k}{2^n}, \frac{k+1}{2^n}]}|f(s)-f(t)|>\varepsilon}
 \mathbbm{1}_{\sup\limits_{s,t \in [\frac{k}{2^n}, \frac{k+1}{2^n}]}|g(s)-g(t)|>\varepsilon}
 =0.
 $$
 \end{lem}
 \begin{proof}
  Согласно лемме~\ref{boundedJumpsNumberLem}, при всех достаточно больших $n$ ($n \ge n_0$)
 неравенства
  $$\sup\limits_{s,t \in [\frac{k}{2^n}, \frac{k+1}{2^n}]}|f(s)-f(t)|>\varepsilon,$$
  $$\sup\limits_{s,t \in [\frac{k}{2^n}, \frac{k+1}{2^n}]}|g(s)-g(t)|>\varepsilon$$
  могут выполняться лишь для отрезков $[\frac{k}{2^n}, \frac{k+1}{2^n}]$,
  содержащих разрывы $f, g$ соответственно величины $\ge \varepsilon.$
  Поскольку разрывов $f$, $g$ величины $\ge \varepsilon$ конечное число, и все они находятся в разных
  точках, то при достаточно больших $n$ они будут разделены отрезками разбиения, и при достаточно больших $n$
  никакие два таких разрыва не придутся на один отрезок вида 
  $[\frac{k}{2^n}, \frac{k+1}{2^n}]$.
 \end{proof}
 
 \begin{lem}\label{sumOverJumpsLem}
  Пусть $f, g \in \mathbb{D}[0,1]$ не имеют общих моментов скачков, $\varepsilon>0$
  фиксировано. Тогда
 $$\sum\limits_{k=0}^{2^n-1}
 \mathbbm{1}_{\sup\limits_{s,t \in [\frac{k}{2^n}, \frac{k+1}{2^n}]}|f(s)-f(t)|>\varepsilon}
 \left|g\left(\frac{k+1}{2^n}\right)-g\left(\frac{k}{2^n}\right)\right|\xrightarrow[n\to\infty]{}0.
 =0.
 $$
 \end{lem}
 \begin{proof}
  По лемме~\ref{boundedJumpsNumberLem}, для некоторого $K > 0$ при всех достаточно больших $n$
 в интересующей нас сумме не болеее $K$ слагаемых. 
 С другой стороны, по лемме~\ref{simultaneousJumpsLem}, при всех достаточно больших $n$
 на всех отрезках $[\frac{k}{2^n}, \frac{k+1}{2^n}]$, которые дают вклад в сумму, 
 имеет место неравенство 
 $\left|g\left(\frac{k+1}{2^n}\right)-g\left(\frac{k}{2^n}\right)\right| \le \varepsilon.$
 Отсюда следует, что наша сумма при достаточно больших $n$ не превосходит $K\varepsilon.$
 Однако, это ещё не доказывает лемму. 
 
 Поэтому усложним наше рассуждение. Выберем произвольное $\delta, 0 < \delta < \varepsilon.$
 При достаточно больших $n$ все ненулевые слагаемые в сумме не превосходят $\delta$
 в силу леммы~\ref{simultaneousJumpsLem}, а количество ненулевых слагаемых не более $K$ (константа 
 $K$ та же, что в начале доказательства).
 Следовательно, при достаточно больших $n$ наша сумма не превосходит $K\delta.$
 Поскольку $\delta$ произвольное, а $K$ фиксировано, то лемма доказана. 
 \end{proof}

  \begin{lem}
  Пусть $f, g \in \mathbb{D}[0,1]$ --- неубывающие функции без общих
  моментов скачков, $h\in C[0,\infty)$ --- непрерывная функция. Тогда
 $$S_n=\sum\limits_{k=0}^{2^n-1}
 \sup\limits_{s,t \in [f(\frac{k}{2^n}), f(\frac{k+1}{2^n})]}(h(s)-h(t))^2
 \left|g\left(\frac{k+1}{2^n}\right)-g\left(\frac{k}{2^n}\right)\right|\xrightarrow[n\to\infty]{}0. 
 $$
 \end{lem}
 \begin{proof}
 Фиксируем $\varepsilon>0$. Выберем $\delta>0$ так, что при любых $s,t \in [f(0), f(1)]$,
 для которых $|s-t| \le \delta$, имеет место неравенство $|h(s)-h(t)| \le \varepsilon$.
 Этот выбор возмолжен в силу непрерывности $h$. 
 Разобъём интересующую нас сумму $S_n$ на две части:
 $$S_n=G_n+H_n.$$
 К $G_n$ отнесём те слагаемые, в которых 
 $f(\frac{k+1}{2^n})-f(\frac{k}{2^n}) > \delta.$
 К $H_n$ отнесём те слагаемые, в которых 
 $f(\frac{k+1}{2^n})-f(\frac{k}{2^n}) \le \delta.$
 По лемме~\ref{sumOverJumpsLem}, учитывая ограниченность функции $h$
 на отрезке $[f(0), f(1)]$, получаем $G_n \to 0 (n \to \infty).$
 С другой стороны, ясно, что для $H_n$ выполнена оценка
 $H_n \le \varepsilon^2(g(1)-g(0)).$
 Следовательно,
 $$\varlimsup\limits_{n\to \infty}S_n \le \varepsilon^2(g(1)-g(0)).$$
 В силу произвольности $\varepsilon$ получаем 
 $\varlimsup\limits_{n\to \infty}S_n = 0$, что и завершает доказательство леммы.
 \end{proof}
 
 \begin{definition}
 Будем говорить, что семейство случайных величин $\gamma_{n,T}$
 стохастически мало на бесконечности, если
  $$\forall \varepsilon>0 \enskip \exists n_0 \enskip
    \forall n \ge n_0 \enskip \exists T_0(n)\enskip \forall T \ge T_0(n) \colon
    P(\gamma_{n, T}>\varepsilon)<\varepsilon.$$
 \end{definition}

  \begin{definition}
 Будем говорить, что семейство случайных величин $\gamma_{n,T}$
 стохастически ограничено на бесконечности, если
  $$\forall \varepsilon>0 \enskip \exists C>0 \enskip \exists n_0 \enskip
    \forall n \ge n_0 \enskip \exists T_0(n)\enskip \forall T \ge T_0(n) \colon
    P(\gamma_{n, T}>C)<\varepsilon.$$
 \end{definition}
 
 \begin{lem}\label{simpleLemAboutStochasticSmallness}
  Пусть $\xi_{n,T}$ стохастически мало на бесконечноости,
  $\eta_{n,T}$ стохастически ограничено на бесконечности.
  Тогда $\zeta_{n,T}=\eta_{n,T}\xi_{n,T}$ стохастически мало
  на бесконечности. 
 \end{lem}
 \begin{proof}
   Фиксируем $\varepsilon>0$. Выберем $C, n_1, n_2$ так, что
   $$\forall n \ge n_1 \enskip \exists T_1(n)\enskip \forall T \ge T_1(n) \colon
    P(\xi_{n, T}>\varepsilon)<\varepsilon.$$
  
  $$\forall n \ge n_2 \enskip \exists T_2(n)\enskip \forall T \ge T_2(n) \colon
    P(\eta_{n, T}>C)<\varepsilon.$$
    Тогда при всех $n \ge n_0=\max\{n_1, n_2\}, T \ge T_0(n)=\max\{T_1(n), T_2(n)\}$ имеет место 
  $$P(\zeta_{n, T}>C\varepsilon)<2\varepsilon.$$
  Лемма доказана. 
 \end{proof}
 
 \begin{lem}\label{anyALemma}
  Если $\varlimsup\limits_{T\to \infty}\xi_T \le \xi \mbox{ п.н.}$, то при любом $A$
 $$\varlimsup\limits_{T \to \infty}P(\xi_T \ge A) \le P(\xi \ge A).$$
 \end{lem}
 \begin{proof}
  Если $\varlimsup\limits_{T\to \infty}a_T \le a$, то
 $$\varlimsup\limits_{T\to\infty}\mathbbm{1}_{a_T \ge A} \le \mathbbm{1}_{a \ge A}.$$
 Итак, 
 $$\varlimsup\limits_{T\to\infty}\mathbbm{1}_{\xi_T \ge A} \le \mathbbm{1}_{\xi \ge A}.$$
 В силу леммы Фату,
 $$P(\xi \ge A)=\mathbb{E}\mathbbm{1}_{\xi \ge A} \ge \mathbb{E}\varlimsup\limits_{T\to \infty}\mathbbm{1}_{\xi_T \ge A} \ge
 \varlimsup\limits_{T\to \infty}\mathbb{E}\mathbbm{1}_{\xi_T \ge A} =
 \varlimsup\limits_{T \to \infty}P(\xi_T \ge A).$$
 \end{proof}

 \begin{lem}\label{sumOverJumpsStochasticLem}
  Пусть $(\xi^{(T)}, \eta^{(T)})\xrightarrow[t\to\infty]{fd} (\xi, \eta),$
  где $\xi, \eta$ принимают значения в пространстве $\mathbb{D}[0,1]$
  и с вероятностью $1$ не имеют общих моментов скачка. 
  Пусть $\varepsilon_0>0$ фиксировано,
  $$\gamma_{n,T}=\sum\limits_{k=0}^{2^n-1}
  \mathbbm{1}_{\sup\limits_{s, t \in [\frac{k}{2^n}, \frac{k+1}{2^n}]}
  |\xi^{(T)}(t)-\xi^{(T)}(s)|>\varepsilon_0}
  \left|\eta^{(T)}\left(\frac{k+1}{2^n}\right)-\eta^{(T)}\left(\frac{k}{2^n}\right)\right|.$$
  Тогда $\gamma_{n,T}$ --- стохастически мало на бесконечности. 
 \end{lem}
 \begin{proof}
  Фиксируем $\varepsilon>0.$
  Пусть 
  $$\gamma_n=\sum\limits_{k=0}^{2^n-1}
  \mathbbm{1}_{|\xi(\frac{k+1}{2^n})-\xi(\frac{k}{2^n})| \ge \varepsilon_0/2}
  \left|\eta\left(\frac{k+1}{2^n}\right)-\eta\left(\frac{k}{2^n}\right)\right|.$$
  Тогда $\gamma_n\to 0$ п.н. в силу леммы~\ref{sumOverJumpsLem}.
  Выберем $n_0$ так, что при $n \ge n_0$ $P(\gamma_n \ge \varepsilon)<\varepsilon.$
  Теперь при $n \ge n_0$ рассуждаем так:
  для $(\xi^{(T)}(\frac{k}{2^n}), \eta^{(T)}(\frac{k}{2^n}), 0 \le k \le 2^n-1)$
  и $(\xi(\frac{k}{2^n}), \eta(\frac{k}{2^n}), 0 \le k \le 2^n-1)$
  существует представление Скорохода, реализующее сходимость п.н.:
  $$\left(\xi^{(T)}\left(\frac{k}{2^n}\right), \eta^{(T)}\left(\frac{k}{2^n}\right), 
  0 \le k \le 2^n-1\right) \stackrel{d}{=}
  \left(\tilde{\xi}^{(T)}\left(\frac{k}{2^n}\right), \tilde{\eta}^{(T)}\left(\frac{k}{2^n}\right), 
  0 \le k \le 2^n-1\right),$$
  $$\left(\xi\left(\frac{k}{2^n}\right), \eta\left(\frac{k}{2^n}\right), 
  0 \le k \le 2^n-1\right) \stackrel{d}{=}
  \left(\tilde{\xi}\left(\frac{k}{2^n}\right), \tilde{\eta}\left(\frac{k}{2^n}\right), 
  0 \le k \le 2^n-1\right),$$
  и при фиксированном $n$
  $$\left(\tilde{\xi}^{(T)}\left(\frac{k}{2^n}\right), \tilde{\eta}^{(T)}\left(\frac{k}{2^n}\right), 
  0 \le k \le 2^n-1\right) 
  \xrightarrow[T\to\infty]{}
  \left(\tilde{\xi}\left(\frac{k}{2^n}\right), \tilde{\eta}\left(\frac{k}{2^n}\right), 
  0 \le k \le 2^n-1\right) \mbox{п.н.}$$
  Значит, при фиксированном $n \ge n_0$
  $$\varlimsup\limits_{T \to \infty}\tilde{\gamma}_{n, T} \le \tilde{\gamma}_n,$$
  и потому в силу леммы~\ref{anyALemma}
  $$\varlimsup\limits_{T \to \infty}P(\gamma_{n, T} \ge \varepsilon)\le P(\gamma_n \ge \varepsilon).$$
  Следовательно, для данного фиксированного $n \ge n_0$ имеем при достаточно больших $T$:
  $$P(\gamma_{n,T}>2\varepsilon) \le P(\gamma_n \ge \varepsilon)<\varepsilon.$$
  Это и доказывает лемму.
 \end{proof}

 \begin{lem}\label{mostSubtleGammaNTLem}
  Пусть $\xi^{(T)}, \eta^{(T)}$ --- неубывающие неотрицательные процессы,
  $(\xi^{(T)}, \eta^{(T)})\xrightarrow[T\to\infty]{fd} (\xi, \eta),$
  где $\xi, \eta$ принимают значения в пространстве $\mathbb{D}[0,1]$
  и с вероятностью $1$ не имеют общих моментов скачка. $\beta_{n, T}$ ---винеровские
  процессы. 
  $$\gamma_{n,T}=\sum\limits_{k=0}^{2^n-1}
  \sup\limits_{s,t \in [\xi^{(T)}(\frac{k}{2^n}), \xi^{(T)}(\frac{k+1}{2^n})]}
  \left|\beta_{n,T}(t)-\beta_{n,T}(s)\right|^2
  \left(\eta^{(T)}\left(\frac{k+1}{2^n}\right)-\eta^{(T)}\left(\frac{k}{2^n}\right)\right).$$
  Тогда $\gamma_{n,T}$ стохастически мало на бесконечности. 
 \end{lem}
 \begin{proof}
 Фиксируем $\varepsilon>0$. 
 Выберем $C>0$ так, что 
 $$P\left(\xi^{(T)}(1)>C\right)<\varepsilon, T>T_0.$$
 Выберем $\delta>0$ так, что для винеровского процесса $\beta$
 $$P\left(\sup\limits_{s, t \in [0, C], |s-t| \le \delta}|\beta(t)-\beta(s)|>\varepsilon\right) <\varepsilon.$$ 
 Обозначим 
 $$G_{n, T} = \sum\limits_{k=0}^{2^n-1} \mathbbm{1}_{\xi^{(T)}(\frac{k+1}{2^n})-\xi^{(T)}(\frac{k}{2^n})>\delta}
  \sup\limits_{s,t \in [\xi^{(T)}(\frac{k}{2^n}), \xi^{(T)}(\frac{k+1}{2^n})]}  
  \left|\beta_{n,T}(t)-\beta_{n,T}(s)\right|^2
  \left(\eta^{(T)}\left(\frac{k+1}{2^n}\right)-\eta^{(T)}\left(\frac{k}{2^n}\right)\right),$$
 
 $$H_{n, T} = \sum\limits_{k=0}^{2^n-1}
  \mathbbm{1}_{\xi^{(T)}(\frac{k+1}{2^n})-\xi^{(T)}(\frac{k}{2^n}) \le \delta}
  \sup\limits_{s,t \in [\xi^{(T)}(\frac{k}{2^n}), \xi^{(T)}(\frac{k+1}{2^n})]}  
  \left|\beta_{n,T}(t)-\beta_{n,T}(s)\right|^2
  \left(\eta^{(T)}\left(\frac{k+1}{2^n}\right)-\eta^{(T)}\left(\frac{k}{2^n}\right)\right).$$
  
  Заметим, что 
  $$ G_{n,T} \le 
   4 \sup\limits_{t \in [0, \xi^{(T)}(1)]}
  \left|\beta_{n,T}(t)\right|^2
  \sum\limits_{k=0}^{2^n-1} \mathbbm{1}_{\xi^{(T)}(\frac{k+1}{2^n})-\xi^{(T)}(\frac{k}{2^n})>\delta}
  \left(\eta^{(T)}\left(\frac{k+1}{2^n}\right)-\eta^{(T)}\left(\frac{k}{2^n}\right)\right).
 $$  
 Тогда $G_{n, T}$ стохастически мало на бесконечности
 в силу лемм~\ref{simpleLemAboutStochasticSmallness},
 ~\ref{sumOverJumpsStochasticLem} и стохастической ограниченности на бесконечности
 $\sup\limits_{t \in [0, \xi^{(T)}(1)]}\left|\beta_{n,T}(t)\right|^2$.
 Для $H_{n, T}$ имеем при $T>T_0$: 
 \begin{multline*}
   P\left(H_{n, T} > \varepsilon^2 \left(\eta^{(T)}(1)-\eta^{(T)}(0)\right)\right)\le 
  P\left(\xi^{(T)}(1)>C\right)+\\+
  P\left(\sup\limits_{s, t \in [0, C], |s-t| \le \delta}|\beta_{n,T}(t)-\beta_{n,T}(s)|>\varepsilon\right) <
  2\varepsilon.
 \end{multline*}
  Поэтому и $\gamma_{n,T}=G_{n,T}+H_{n,T}$ стохастически мало на бесконечности в силу произвольности $\varepsilon.$ 
 \end{proof}
 
 
 \begin{lem}\label{mostSubtleCharacteristicsGammaNTLem}
  Пусть $\eta^{(T)}$ --- неубывающие неотрицательные процессы,
  $(\xi^{(T)}, \eta^{(T)})\xrightarrow[t\to\infty]{fd} (\xi, \eta),$
  где $\xi, \eta$ принимают значения в пространстве $\mathbb{D}[0,1]$
  и с вероятностью $1$ не имеют общих моментов скачка,  
  $$\gamma_{n,T}=\sum\limits_{k=0}^{2^n-1}
  \sup\limits_{s,t \in [\frac{k}{2^n}), \frac{k+1}{2^n}]}
  \left|\xi^{(T)}(t)-\xi^{(T)}(s)\right|
  \left(\eta^{(T)}\left(\frac{k+1}{2^n}\right)-\eta^{(T)}\left(\frac{k}{2^n}\right)\right).$$
    Тогда $\gamma_{n,T}$ стохастически мало на бесконечности. 
 \end{lem}
 \begin{proof}  
 Фиксируем $\varepsilon>0$. Обозначим 
 $$G_{n, T} = \sum\limits_{k=0}^{2^n-1} \mathbbm{1}_{\exists s, t \in [\frac{k}{2^n}), \frac{k+1}{2^n}] \colon
 |\xi(t)-\xi(s)| \ge \varepsilon}
  \sup\limits_{s,t \in [\frac{k}{2^n}, \frac{k+1}{2^n}]}  
   \left|\xi^{(T)}(t)-\xi^{T}(s)\right|^2
  \left(\eta^{(T)}\left(\frac{k+1}{2^n}\right)-\eta^{(T)}\left(\frac{k}{2^n}\right)\right),$$
 
 $$H_{n, T} = \sum\limits_{k=0}^{2^n-1}
  \mathbbm{1}_{\forall s, t \in [\frac{k}{2^n}), \frac{k+1}{2^n}] \colon |\xi(t)-\xi(s)| \le \varepsilon}
  \sup\limits_{s,t \in [\frac{k}{2^n}, \frac{k+1}{2^n}]}  
  \left|\xi^{(T)}(t)-\xi^{(T)}(s)\right|^2
  \left(\eta^{(T)}\left(\frac{k+1}{2^n}\right)-\eta^{(T)}\left(\frac{k}{2^n}\right)\right).$$
  
  Тогда $G_{n, T}$ стохастически мало на бесконечности
  в силу лемм~\ref{simpleLemAboutStochasticSmallness},
  ~\ref{sumOverJumpsStochasticLem} 
  и стохастической ограниченности $\sup\limits_{s\in [0,1]}|\xi^{(T)}(s)|.$
  
  Для $H_{n, T}$ имеем при $T>T_0$
  $$
   H_{n, T} \le \varepsilon^2 \left(\eta^{(T)}(1)-\eta^{(T)}(0)\right).
  $$
  Поэтому и $\gamma_{n,T}=G_{n,T}+H_{n,T}$ стохастически мало на бесконечности
  в силу произвольности $\varepsilon.$ 
 \end{proof}

\begin{lem}\label{integralDominatedConvergenceLem}
Пусть $\xi$ --- случайный процесс со значениями в $\mathbb{D}[0,1]$
без фиксированных моментов скачка, 
$\eta$ --- семимартингал с траекториями в $\mathbb{D}[0,1]$. 
Тогда сумма
$$\sum\limits_{k=0}^{n-1}\xi\left(\frac{k}{n}\right)\left(\eta\left(\frac{k+1}{n}\right)-
\eta\left(\frac{k}{n}\right)\right)$$
сходится по вероятности к $\int\limits_0^1 \xi(s-)d\eta(s).$
\end{lem}
\begin{proof}
 Эта сходимость выполняется в силу теоремы об ограниченной сходимости для стохастических
 интегралов (см.\cite{Meyer}, глава~8).
Поясним более детально. Согласно~\cite{Kallenberg}, теорема 26.4, выполнено
соотношение:
если $X$ --- семимартингал,
$V_n\to 0$ и $|V_n|\le V$, где все процессы $V_n, V$ --- предсказуемые и локально
ограниченные, то 
$$\sup\limits_{t \in [0,1]}\int\limits_0^t V_n(s)dX(s)\xrightarrow[n\to\infty]{P}0.$$
Положим
$$X(s)=\eta(s),s \ge 0,$$
$$H_n(s)=\xi\left(\frac{[ns]}{n}\right),$$
$$G_n(s)=\xi(s-),$$\
$$V_n(s)=G_n(s)-H_n(s).$$
В силу непрерывности слева процесса $\xi(s-)$, имеем
$$\xi\left(\frac{[ns]}{n}\right)\xrightarrow[n\to\infty]{}\xi(s).$$
(Заметим, что с вероятностью $1$ процесс $\xi$ вообще не имеет разрывов в рациональных
точках, и потому с вероятностью $1$ сходимость имеет место для всех $s$).
Отсюда заключаем, что с вероятностью $1$ имеет место 
$$V_n\to 0, n \to \infty.$$
В роли мажорирующего процесса возьмём
$$V(t)=2\sup\limits_{s\in [0,t]}|\xi(s-)|.$$
Этот процесс является непрерывным слева, а потому предсказуемым и локально ограниченным. 
\end{proof}

 \begin{theorem}\label{firstMainTheorem}
  Пусть $\xi^{(T)}, \eta^{(T)}$ --- 
  непрерывные семимартингалы на $[0,1]$ относительно общей фильтрации, причём
  $$\left(\xi^{(T)}, \eta^{(T)}, \left<\xi^{(T)}\right>,\left<\eta^{(T)}\right>\right)
  \xrightarrow[T\to\infty]{fd} \left(\xi, \eta, \zeta_1, \zeta_2\right),$$
  где $\xi, \eta$ принимают значения в пространстве $\mathbb{D}[0,1]$;
  $\zeta_1, \zeta_2$ принимают значения в пространстве $\mathbb{D}[0,1]$
  и с вероятностью $1$ не имеют общих моментов скачка.
$$\int\limits_0^1 \xi^{(T)}(s)d\eta^{(T)}(s)
\xrightarrow[T \to \infty]{d} \int\limits_0^1 \xi(s-)d\eta(s).$$
 \end{theorem}
 \begin{proof}
  Пусть
$$\gamma_{n, T}=
 \int\limits_0^1 \xi^{(T)}(s)d\eta^{(T)}(s)-
 \sum\limits_{k=0}^{2^n-1}\xi^{(T)}\left(\frac{k}{2^n}\right)\left(\eta^{(T)}
 \left(\frac{k+1}{2^n}\right)-\eta^{(T)}\left(\frac{k}{2^n}\right)\right).$$
Тогда
$$\gamma_{n, T}=
 \sum\limits_{k=0}^{2^n-1}
 \int\limits_{\frac{k}{2^n}}^{\frac{k+1}{2^n}}
 \left(\xi^{(T)}(s)-\xi^{(T)}\left(\frac{k}{2^n}\right)\right)d\eta^{(T)}(s)=
 w_{n,T}\left(\sum\limits_{k=0}^{2^n-1}
 \int\limits_{\frac{k}{2^n}}^{\frac{k+1}{2^n}}
 \left(\xi^{(T)}(s)-\xi^{(T)}\left(\frac{k}{2^n}\right)\right)^2d\left<\eta^{(T)}\right>_s\right),
$$
где $w_{n,T}$ --- некоторые винеровские процессы. 
Пусть $\beta_{n,T}$ --- ассоциированные с $\xi^{(T)}$ винеровские процессы. Имеем
  \begin{multline*}\mu_{n, T}=
\sum\limits_{k=0}^{2^n-1}
 \int\limits_{\frac{k}{2^n}}^{\frac{k+1}{2^n}}
 \left(\xi^{(T)}(s)-\xi^{(T)}\left(\frac{k}{2^n}\right)\right)^2d\left<\eta^{(T)}\right>_s \le \\ \le
\sum\limits_{k=0}^{2^n-1}
 \sup\limits_{s, t \in [\left<\xi^{(T)}\right>(\frac{k}{2^n}), \left<\xi^{(T)}\right>(\frac{k+1}{2^n})]}
  \left|\beta_{n,T}(t)-\beta_{n,T}(s)\right|^2
 \left(\left<\eta^{(T)}\right>\left(\frac{k+1}{2^n}\right)-\left<\eta^{(T)}\right>\left(\frac{k}{2^n}\right)
 \right).
 \end{multline*}
 За счёт леммы~\ref{mostSubtleGammaNTLem} $\mu_{n,T}$ стохастически мало на бесконечности.
 Тогда и $\gamma_{n, T}=w_{n, T}(\mu_{n,T})$ стохастически мало на бесконечности. 
 За счёт лемм~\ref{WeakConvergenceThirdLem}, \ref{integralDominatedConvergenceLem} 
 получаем нужное утверждение. 
 \end{proof}
 
\begin{theorem}\label{secondMainTheorem}
  Пусть 
  $(\xi^{(T)}, \eta^{(T)}, \left<\xi^{(T)}\right>,\left<\eta^{(T)}\right>)
  \xrightarrow[T\to\infty]{fd} (\xi, \eta, \zeta_1, \zeta_2),$
  где $\xi, \eta$ принимают значения в пространстве $\mathbb{D}[0,1]$;
  $\zeta_1, \zeta_2$ принимают значения в пространстве $\mathbb{D}[0,1]$
  и с вероятностью $1$ не имеют общих моментов скачка. Тогда
$$\int\limits_0^1 \xi^{(T)}(s)^2d\left<\eta^{(T)}\right>(s)
\xrightarrow[T \to \infty]{d} \int\limits_0^1 \xi(s-)^2d\left<\zeta_2\right>(s).$$
\end{theorem}

\begin{proof}
Пусть
 \begin{multline*}\gamma_{n, T}=
 \int\limits_0^1 \xi^{(T)}(s)^2d\left<\eta^{(T)}\right>(s)-
 \sum\limits_{k=0}^{2^n-1}\xi^{(T)}\left(\frac{k}{2^n}\right)^2\left(\left<\eta^{(T)}\right>
 \left(\frac{k+1}{2^n}\right)-\left<\eta^{(T)}\right>\left(\frac{k}{2^n}\right)\right)=\\=
 \sum\limits_{k=0}^{2^n-1}
 \int\limits_{\frac{k}{2^n}}^{\frac{k+1}{2^n}}
 \left(\xi^{(T)}(s)^2-\xi^{(T)}\left(\frac{k}{2^n}\right)^2\right)d\left<\eta^{(T)}\right>(s).
 \end{multline*} 
Имеем
$$
|\gamma_{n, T}| \le 
\sum\limits_{k=0}^{2^n-1}
 \sup\limits_{s, t \in [\frac{k}{2^n}, \frac{k+1}{2^n}]}
  \left|\xi^{(T)}(t)^2-\xi^{(T)}(s)^2\right|
 \left(
 \left<\eta^{(T)}\right>\left(\frac{k+1}{2^n}\right)-\left<\eta^{(T)}\right>\left(\frac{k}{2^n}\right)
 \right).
$$
  За счёт леммы~\ref{mostSubtleCharacteristicsGammaNTLem} $\gamma_{n,T}$
стохастически мало на бесконечности.  
 За счёт леммы~\ref{WeakConvergenceThirdLem} получаем нужное утверждение.
\end{proof}

\section{Процессы, порождённые винеровским процессом}
\begin{definition}
Пусть $\beta$ --- винеровский процесс. Семейством процессов $\mathcal U$, порождённым $\beta$, будем 
называть семейство всех процессов вида
$$\int \ldots \int d\eta_1 \ldots d\eta_n,$$
где $d\eta_k(s)=d\beta(s)$ или $d\eta_k(s)=ds$
при каждом $k$.
В частности, процессы
$\xi_1(t)=t,$  
$\xi_2(t)=\beta(t), \xi_3(t)=\int\limits_0^t \beta(s)d\beta(s),$
$\xi_4(t)=\int\limits_0^t \beta(s)d\beta(s),
\xi_5(t)=\int\limits_0^t \xi_2(s) ds$
входят в семейство $\mathcal U$. 
Ясно, что все процессы, порождённые $\beta$, являются
непрерывными семимартингалами относительно фильтрации, 
порождённой $\beta$.
\end{definition}

\begin{comment} 
Будем рассматривать следующее семейство процессов $\mathfrak{A}$, которое будем называть семейством 
процессов, порождённых $\beta.$
\begin{itemize}
 \item Процесс $\beta$ входит в $\mathfrak{A}$;
 \item если процесс $U$ входит в $\mathfrak{A}$, то процесс $V$,
 задаваемый равенством
 $$V_t=\int\limits_0^t U_s ds$$
 входит в $\mathfrak{A}$;
  \item если процесс $U$ входит в $\mathfrak{A}$, то процесс $V$,
 задаваемый равенством
 $$V_t=\int\limits_0^t U_s d\beta_s$$
 входит в $\mathfrak{A}$. 
\end{itemize}
\end{comment}

\begin{comment}
 \begin{remark}
 Все процессы, порождённые процессом $\beta$, являются
 непрерывными семимартингалами относительно фильтрации, 
 порождённой процессом $\beta$.
\end{remark}
\end{comment}

\begin{lem}\label{WienerCalculusLem}
 Если $L$ --- процесс, порождённый процессом $\beta$, то
при любом $n \ge 0$
$$X_t=\int\limits_0^t \beta^n(s)dL_s$$
представляется в виде $X=F(\beta)$,
где $F$ --- некоторое непрерывное отображение
$$F \colon C[0, \infty) \to C[0, \infty).$$
В частности, при $n=0$ получаем, что сам процесс $L$ представляется
в таком виде. 
\end{lem}
\begin{proof}
 Доказательство проведём индукцией по минимальной длине цепочки процессов, порождающей $L$. 
Проверим, что $X$ представляется в виде $F(\beta)$. Два случая
\begin{itemize}
 \item $X_t = \int\limits_0^t Y_s ds$, и $Y$ --- процесс, порождённый $\beta$,
 для которого утверждение леммы доказано. Тогда
$$\int\limits_0^t \beta^n(s)dX_s = \int\limits_0^t \beta^n(s) Y_s ds,$$
и ясно, что это --- непрерывный функционал от $\beta$. 
 \item $X_t = \int\limits_0^t Y_s d\beta(s)$, и для $Y$ утверждение доказано. Тогда
\begin{multline*}
 \int\limits_0^t \beta^n(s)dX_s = \int\limits_0^t \beta^n(s) Y_s d\beta(s)=
 \frac{1}{n}\int\limits_0^t Y_s d\beta^n(s)-\frac{n-1}{2}\int\limits_0^t \beta(s)^{n-2} Y_s ds=\\=
 \frac{1}{n}Y_t \beta^n(t)-\frac{1}{n}\int\limits_0^t \beta^n(s)dY_s-
 \frac{1}{n}\left<Y, \beta^n\right>_t-
 \frac{n-1}{2}\int\limits_0^t \beta(s)^{n-2} Y_s d.
\end{multline*}
Для всех слагаемых, кроме $\frac{1}{n}\left<Y, \beta^n\right>_t$, ясно, что они являются непрерывными 
функционалами от $\beta$. Для $Y$ возможны 2 случая: 
\begin{itemize}
\item $Y_t=\int\limits_0^t Q_s ds,$ и в этом случае 
$\left<Y, \beta^n\right>_t=0$; 
\item $Y_t=\int\limits_0^t Q_s d\beta(s)$,
и в этом случае 
$\frac{1}{n}\left<Y, \beta^n\right>_t=\int\limits_0^t \beta(s)^{n-1}Q_s ds,$
и ясно, что это --- непрерывный функционал от $\beta$. 
\end{itemize}
Лемма доказана. 
\end{itemize}
\end{proof}

\section{Асимптотика инвариантов Васильева}
\begin{definition}
  Пусть $Y_1^{(T)}, \ldots, Y_n^{(T)}$ --- семейства непрерывных семимартингалов
 относительно общей (при каждом фиксированном $T$) фильтрации. 
Назовём совокупностью семейств процессов, порождаемой $Y_1, \ldots, Y_n$,
наименьшую совокупность $S$, такую, что
\begin{itemize}
 \item при любом $k=1, \ldots, n \colon Y_k^{(T)}\in S, \left<Y_k^{(T)}\right> \in S$;
 \item если $U^{(T)}\in S$, то при любом $k$ семейства $V^{(T)}, Q^{(T)}$, определяемые 
равенствами
$$V^{(T)}_t = \int\limits_0^t U^{(T)}(s)dY_k^{(T)}(s), t \ge 0$$
$$Q^{(T)}_t = \int\limits_0^t U^{(T)}(s)d\left<Y_k^{(T)}\right>(s), t \ge 0,$$
лежат в $S$: $V^{(T)}\in S, Q^{(T)}\in S.$
\end{itemize}
\end{definition}

\begin{theorem}\label{YDefinedProcessesTheorem}
 Пусть $Y^{(T)}$ --- семейство непрерывных локальных мартингалов, такое, что
$$Y^{(T)}(t)=\beta^{(T)}\left(\left<Y^{(T)}\right>_t\right),$$
и для любых $t_1, \ldots, t_k$ случайные элементы
$$(\beta^{(T)}, \left<Y^{(T)}\right>(t_1), \ldots, \left<Y^{(T)}\right>(t_k)) 
\in C[0, \infty)\times \mathbb{R}^k$$
сходятся по распределению при $T \to \infty$. 
Пусть $S$ --- совокупность семейств, порождённых $Y^{(T)}$.
Тогда для любого семейства $U^{(T)}\in S$: 
$$(Y^{(T)}, U_1^{(T)}, \ldots,  U_k^{(T)})$$
имеет предел в смысле конечномерных 
распределений:
$$\left(Y^{(T)}, \left<Y^{(T)}\right>,
U_1^{(T)}, \ldots,  U_m^{(T)}\right) \xrightarrow[T \to \infty]{fd}
\left(Y, Q, U_1, \ldots,  U_k\right),$$
и все точки разрыва процессов $Y, U_1, \ldots, U_k$
с вероятностью $1$ лежат среди точек разрыва процесса $Q$. 
\end{theorem}
\begin{proof}
 Из леммы~\ref{WienerCalculusLem} следует, что все процессы
 из указанной совокупности представляются в виде
$$U^{(T)}=F(\beta^{(T)})(\left<Y^{(T)}\right>),$$
где $F\colon C[0,\infty) \to C[0, \infty)$ --- непрерывный функционал от винеровского процесса. 
Отсюда всё сразу следует. 
\end{proof}

\begin{theorem}\label{generalTheorem}
 Пусть $X^{(T)}, Y^{(T)}$ --- непрерывные семимартингалы относительно общей (при каждом фиксированном $T$)
фильтрации, причём
$$(X^{(T)}, Y^{(T)})\xrightarrow[T \to \infty]{d}(X, Y),$$
а процессы $X, Y$ с вероятностью $1$ не имеют общих моментов скачка. 
Пусть также семейство $Y^{(T)}$ удовлетворяет условиям теоремы~\ref{YDefinedProcessesTheorem}.
Обозначим
$$L^{(T)}_0(t) = Y^{(T)}(t),$$
$$L^{(T)}_{k+1}(t) = \int\limits_0^t L_k^{(T)}(s) dY^{(T)}(s).$$
Тогда при любом $k$ для любых семейств $U^{(T)}$ из совокупности, 
порождённой $Y^{(T)}$ (см. теорему~\ref{YDefinedProcessesTheorem}),
конечномерные распределения процессов
$$\int\limits_0^t L_k^{(T)}(s)U^{(T)}(s)dY^{(T)}(s),
\int\limits_0^t L_k^{(T)}(s)U^{(T)}(s)d\left<Y^{(T)}\right>(s)$$
сходятся при $T \to \infty$ совместно с $Y^{(T)}$,
причём все скачки предельных процессов совпадают со скачками процесса $Y.$
\begin{proof}
 Проведём доказательство индукцией по $k$. 
База индукции проверяется за счёт теорем~\ref{firstMainTheorem},
~\ref{secondMainTheorem}.
Шаг индукции: пусть утверждение теоремы проверено для $k$, проверим для $k+1$. 
Обозначим $V^{(T)}(t)=\int\limits_0^t U_k^{(T)}(s)dY_k^{(T)}(s),$
$Q_t=\int\limits_0^t U_k^{(T)}(s)d\left<Y_k^{(T)}\right>(s).$
Ясно, что процессы $V^{(T)}, Q^{(T)}$ имеют пределы в смысле сходимости
конечномерных распределений.
Имеем
$$L^{(T)}_{k+1}(t) = \int\limits_0^t L_k^{(T)}(s) dY^{(T)}(s),$$
\begin{multline*}
 \int\limits_0^t L_{k+1}^{(T)}(s)U^{(T)}(s)dY^{(T)}(s) = \int\limits_0^t L_{k+1}^{(T)}(s) dV^{(T)}(s)=
L_{k+1}^{(T)}(t) V^{(T)}(t)-\\-
\int\limits_0^t L_k^{(T)}(s)V^{(T)}(s)dY^{(T)}(s)-
\left<L_{k+1}^{(T)}, V^{(T)}\right>_t=
\int\limits_0^t L_{k+1}^{(T)}(s) dV^{(T)}(s)=\\=
L_{k+1}^{(T)}(t) V^{(T)}(t)-
\int\limits_0^t L_k^{(T)}(s)V^{(T)}(s)dY^{(T)}(s)-
\int\limits_0^t L_k^{(T)}(s)U^{(T)}(s)d\left<Y^{(T)}\right>_s,
\end{multline*}
и по предположению индукции имеем сходимость конечномерных распределений для 
$$\int\limits_0^t L_{k+1}^{(T)}(s)U^{(T)}(s)dY^{(T)}(s).$$
Далее, 
\begin{multline*}
 \int\limits_0^t L_{k+1}^{(T)}(s)U^{(T)}(s)d\left<Y^{(T)}\right>(s) =
\int\limits_0^t L_{k+1}^{(T)}(s) dQ^{(T)}(s)=
L_{k+1}^{(T)}(t) Q^{(T)}(t)-\\-
\int\limits_0^t L_k^{(T)}(s)Q^{(T)}(s)dY^{(T)}(s)-
\left<L_{k+1}^{(T)}, V^{(T)}\right>_t=
\int\limits_0^t L_{k+1}^{(T)}(s) dV^{(T)}(s)=\\=
L_{k+1}^{(T)}(t) V^{(T)}(t)-
\int\limits_0^t L_k^{(T)}(s)V^{(T)}(s)dY^{(T)}(s)-
\int\limits_0^t L_k^{(T)}(s)U^{(T)}(s)d\left<Y^{(T)}\right>_s,
\end{multline*}
и по предположению индукции имеем сходимость конечномерных распределений для 
$$\int\limits_0^t L_{k+1}^{(T)}(s)U^{(T)}(s)d\left<Y^{(T)}\right>(s).$$
\end{proof}
\end{theorem}

\begin{theorem}\label{MainTheoremForWienerProcesses}
 Пусть $Y_1^{(T)}, \ldots, Y_n^{(T)}$ --- семейства непрерывных локальных мартингалов относительно общей
(при каждом фиксированном $T$) фильтрации, такие, что 
$$(Y_1^{(T)}, \ldots, Y_n^{(T)})\xrightarrow[T \to \infty]{fd} (Y_1, \ldots, Y_n),$$
причём с вероятностью $1$ процессы
$Y_1, \ldots, Y_n$
попарно не имеют общих моментов скачка, и
при любом $k$ выполнено хотя бы одно из двух условий
\begin{itemize}
 \item семейство $Y_k^{(T)}$ удовлетворяет условиям 
теоремы~\ref{YDefinedProcessesTheorem};
 \item предельный процесс $Y_k$ 
не имеет точек разрыва с вероятностью $1$,
а характеристики $\left<Y_k^{(T)}\right>$ совпадают с характеристиками
$\left<Y_l^{(T)}\right>$ при некотором $l \ne k$, причём
семейство $Y_l^{(T)}$ удовлетворяет условиям 
теоремы~\ref{YDefinedProcessesTheorem}. 
\end{itemize}
Пусть $S$ --- совокупность семейств процессов, порождаемая $Y_1, \ldots, Y_n$.
Тогда для любых $X_1, \ldots, X_k \in S$ случайный вектор
$$(X_1^{(T)}, \ldots, X_k^{(T)})$$
сходится при $T \to \infty$ в смысле сходимости конечномерных распределений. 
\end{theorem}
\begin{proof}
 Следует из теоремы~\ref{generalTheorem}.  
\end{proof}

Пусть $Z_k(t), t \ge 1$ --- двумерные броуновские движения, выходящее из попарно различных
точек плоскости. Пусть $Z_{kl}(t)=\|Z_k(t)-Z_l(t)\|, R_{kl}(t)=|Z_{kl}(t)|$,
$\theta_{kl}$ --- угол обхода процесса $Z_{kl}$ вокруг начала координат 
к моменту времени $t$, $\theta_{kl}(1)=0.$
Введём, следуя~\cite{BertoinWerner}, процессы $X_{kl}$ с помощью экспоненциальной замены времени
для $Z_{kl}$:
$$X_{kl}(t)=e^{-t/2}Z_{kl}(e^t), t \ge 0.$$
Тогда $X_{kl}(t)$ --- двумерный процесс Орнштейна-Уленбека. 
Пусть $\alpha_{kl}(t)=\theta_{kl}(e^t)$ --- угол обхода процесса $X_{kl}$ вокруг начала координат 
к моменту времени $t$. Пусть 
$$\phi_{kl}^{(T)}(t)=\frac{\alpha_{kl}(tT)}{T/2}, 
r_{kl}^{(T)}=\frac{\ln{R_{kl}(e^{tT})}}{T/2}, T>0.$$
Имеют место сходимости в смысле конечномерных распределений
$$\phi_{kl}^{(T)}\xrightarrow[T \to \infty]{fd}\xi, r_{kl}^{(T)}\xrightarrow[T \to \infty]{fd}\eta,$$
где $\xi$ --- процесс Коши, $\eta(t)=t.$
Применим теорему~\ref{MainTheoremForWienerProcesses} к семействам непрерывных локальных мартингалов
$\phi_{kl}^{(T)}, r_{kl}^{(T)}.$

\begin{theorem}
 Пусть $\mathfrak{S}$ --- совокупность семейств процессов, порождаемая 
$$\phi_{kl}^{(T)}, r_{kl}^{(T)}, 1 \le k < l \le n,$$
Тогда для любых $X_1, \ldots, X_k \in \mathfrak{S}$ случайный вектор
$$(X_1^{(T)}, \ldots, X_k^{(T)})$$
сходится при $T \to \infty$ в смысле сходимости конечномерных распределений
\end{theorem}
 
\section{Асимптотика инварианта второго порядка c идентификацией предельного процесса}
Воспользуемся результатом предыдущего раздела для идентификации предельного распределения.

Пусть $\theta(t)$ --- угол обхода вокруг начала координат 
броуновского движения, выходящего из точки $(1, 0)$ в момент $1$,
к моменту времени $t$, $\theta(1)=0.$
Обозначим 
$$\alpha(t)=\theta(e^t),
\phi_T(t)=\frac{\alpha(tT)}{T}.$$

Мы знаем, что конечномерные распределения процесса $\phi_T$ сходятся при $T\to\infty$
к конечномерным распределениям процесса Коши $C$:
$$\phi_T\xrightarrow[T\to\infty]{fd}C.$$

Покажем теперь, что для независимых процессов $\phi^{(1)}, \phi^{(2)}$
имеет место сходимость конечномерных распределений 
$$\int\limits_0^t \phi^{(1)}_T(s)d\phi^{(2)}_T(s)\xrightarrow[T\to\infty]{fd}
\int\limits_0^t C_1(s-)dC_2(s),$$
где $C_1, C_2$ --- независимые процессы Коши.

Для начала проверим, что интеграл
$$\int\limits_0^t C_1(s-)dC_2(s)$$
корректно определён. 

Согласно~\cite{Kallenberg}, соответствующий интеграл определён, если 
$C_1(s-)$ является локально ограниченным процессом, а
$C_2$ --- семимартиннгал. 
Проверим, что это так. 

\begin{statement}
 Процесс $X(s)=C(s-)$, где $C$ --- процесс Коши, является локально ограниченным.
\end{statement}
\begin{proof}
 Нам нужно проверить существование последовательности моментов остановки
 $\tau_n\to\infty$, такой, что процессы $X^{\tau_n}$
 являются ограниченными.
 Выберем 
 $$\tau_n=\inf\{s\colon |X(s)|>n\}.$$
 Ясно, что $|X^{\tau_n}|\le n$, поскольку процесс $X$ непрерывен слева. 
 
 Более общо, можно сказать, что любой непрерывный слева ограниченный с вероятностью $1$
 на каждом конечном интервале случайный процесс является локально ограниченным.  
\end{proof}

\begin{statement}
 Процесс Коши $C(s)$ является семимартингалом. 
\end{statement}
\begin{proof}
 Согласно~\cite{Kallenberg}, семимартингалом является непрерывный справа согласованный
случайный процесс $X$, допускающий представление 
$$X=M+A,$$
где $M$ --- локальный мартингал, $A$ --- процесс локально конечной вариации, $A(0)=0$. 
Выделим из процесса Коши $C$ все скачки величины более $1$ и объединим их в процесс $A$. 
Тогда $C-A$ будет локальным мартингалом с ограниченными скачками, и потому даже локально
квадратично интегрируемым мартингалом. 
 Процесс $A$ имеет конечно много скачков на каждом конечном интервале, поэтому, взяв
 моменты остановки $\tau_n=n$, получим: $A^{\tau_n}$ --- имеет локально конечную вариацию
(мы не говорим ``локально ограниченную''!)
\end{proof}

\begin{statement}\label{integralDominatedConvergenceStatement}
 Сумма
$$\sum\limits_{k=0}^{n-1}C_1\left(\frac{k}{n}\right)\left(C_2\left(\frac{k+1}{n}\right)-
C_2\left(\frac{k}{n}\right)\right)$$
сходится по вероятности к $\int\limits_0^1 C_1(s-)dC_2(s).$
\end{statement}
\begin{proof}
 Эта сходимость выполняется в силу теоремы об ограниченной сходимости для стохастических
 интегралов (см. Мейер).
Поясним более детально. Согласно~\cite{Kallenberg}, теорема 26.4, выполнено
соотношение:
если $X$ --- семимартингал,
$V_n\to 0$ и $|V_n|\le V$, где все процессы $V_n, V$ --- предсказуемые и локально
ограниченные, то 
$$\sup\limits_{t \in [0,1]}\int\limits_0^t V_n(s)dX(s)\xrightarrow[n\to\infty]{P}0.$$
Положим
$$X(s)=C_2(s),s \ge 0,$$
$$H_n(s)=C_1\left(\frac{[ns]}{n}\right),$$
$$G_n(s)=C_1(s-),$$\
$$V_n(s)=G_n(s)-H_n(s).$$
В силу непрерывности слева процесса $C_1(s-)$, имеем
$$C_1\left(\frac{[ns]}{n}\right)\xrightarrow[n\to\infty]{}C_1(s).$$
(Заметим, что с вероятностью $1$ процесс $C_1$ вообще не имеет разрывов в рациональных
точках, и потому с вероятностью $1$ сходимость имеет место для всех $s$).
Отсюда заключаем, что с вероятнсотью $1$ имеет место 
$$V_n\to 0 (n \to \infty).$$
В роли мажорирующего процесса возьмём
$$V(t)=2\sup\limits_{s\in [0,t]}|C_1(s-)|.$$
Этот процесс является непрерывным слева, а потому предсказуемым и локально ограниченным. 
\end{proof}

\begin{statement}
 Имеет место сходимость
$$\int\limits_0^1 \phi^{(1)}_T(s)d\phi^{(2)}_T(s)
\xrightarrow[T\to\infty]{d} 
\int\limits_0^1 C_1(s-)dC_2(s).$$
\end{statement}

\begin{proof}
 Применим лемму~\ref{WeakConvergenceSecondLem}. Положим
$$\xi(T)=\int\limits_0^1 \phi^{(1)}_T(s)d\phi^{(2)}_T(s),$$
$$\eta_n(T)=
\sum\limits_{k=0}^{n-1}\phi^{(1)}_T\left(\frac{k}{n}\right)\left(\phi^{(2)}_T\left(\frac{k+1}{n}\right)-
\phi^{(2)}_T\left(\frac{k}{n}\right)\right),$$
$$\eta_n=\sum\limits_{k=0}^{n-1}C_1\left(\frac{k}{n}\right)\left(C_2\left(\frac{k+1}{n}\right)-
C_2\left(\frac{k}{n}\right)\right).$$

По лемме~\ref{WeakConvergenceFifthLem} получаем, что 
$\xi(T)\xrightarrow[T\to\infty]{d}\xi_0$, и (что позволяет нам идентифицировать предел)
$\eta_n\xrightarrow[n\to\infty]{d}\xi_0.$

По утверждению~\ref{integralDominatedConvergenceStatement}, получаем
$$\eta_n\xrightarrow[n\to\infty]{d}\int\limits_0^1 C_1(s-)dC_2(s).$$
Таким образом,
$$\xi(T)\xrightarrow[T\to\infty]{d} \int\limits_0^1 C_1(s-)dC_2(s).$$
\end{proof}

\begin{thebibliography}{99}
\bibitem{Kallenberg}
Kallenberg~O.
Foundations of modern probability. --- Springer, 2002. --- 638~p.

\bibitem{Meyer}
Dellacherie, Meyer, J. P.
Probabilities and potential B. Theory of martingales. --- 
North-Holland Publishing Company, 1982. --- 482~p. 

\bibitem{BertoinWerner}
Jean Bertoin, Wendelin Werner.
Asymptotic windings of planar brownian motion revisited via the Ornstein-Uhlenbeck process //
Seminaire de probabilites de Strasbourg (1994),Vol.~28, pp.~138--152. 

\end{thebibliography}

\end {document}

